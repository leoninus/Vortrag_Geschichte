\documentclass{article}
\usepackage[ngerman]{babel}
\usepackage[utf8]{inputenc}


\begin{document}

\title{Zusammenfassung Kreta}
\author{Sontje Schärer, Leanne Fehr, Leonin Hallström}

\maketitle
Kreta ist die grösste griechische Insel und die insgesamt fünftgrösste Insel im Mittelmeer (nach Korsika, Sardinien, Sizilien, Zypern). Auf den
insgesamt 8261 Quadratkilometern wohnen insgesamt 662913 Einwohner (das sind etwa so viele Einwohner wie im Kanton Aargau, der fast sechsmal keiner!
ist). Die grösste Stadt auf Kreta ist Iraklio mit 173450 Einwohnern (das sind etwa so viele Einwohner wie in Basel). In Kreta hat es eher weniger
verschiedene Tierarten, man sieht dort vor allem Ziegen und Schafe. Die Insel gehört erst seit 1913 mit Griechenland und wurde im zweiten Weltkrieg als Truppenstützpunkt
von den Allierten benutzt, um die Schifffahrt im Mittelmeer zu sichern. 

Es gibt mehrere Mythen, die sich um die Insel Kreta ranken, Zeus Geburtshöhle
war beispielsweise auf Kreta. Doch der grosse Mythos von Kreta handelte um Minos. Dieser war ein Sohn von Zeus und Europa, der auf der Insel Kreta
wohnte. Er bat seinen Onkel Poseidon, ihm ein Wunder zu schicken, damit seine Königswürde gestärkt wäre. Er versprach, was immer dem Meer entsteige,
dem Gott zu opfern. Poseidon sndte ihm einen Stier und Minos wurde König von Kreta. Doch Minos gefiel der Stier so gut, dass er ihn in seine Herde
aufnahm und stattdessen ein minderwertiges Tier opferte. Poseidon wurde wütend und schlug Minos' Frau Pasiphaë vor, sich in einem hölzernen Gestell
mit Kuhhaut zu verstecken und sich mit dem Stier zu paaren. Und so entstand der Minotaurus, halb Stier halb Mensch. Minos liess den genialen Erfinder
Dädalus (ein Sohn von Athene) ein Gefängnis in Form eines Labyrinths bauen, in diesem er den Minotaurus gefangen hielt. Doch dann erhielt Minos die
Nachricht, dass sein Sohn durch verschulden des Athener König gestorben sei. Er brach zu einem Rachefeldzug auf, besiegte die Athener und erlegte
ihnen einen grausamen Tribut auf: Jede neun Jahre mussten 14 Jugendliche (7 Knaben, 7 Mädchen) nach Kreta gehen, wo sie in das Labyrinth gingen und so
dem Minotaurus geopfert wurden. Schliesslich löste Theseus das Problem, indem er sich selbst mit der dritten Tributfahrt auf den Weg macht und das
Ungeheuer tötete. Doch Minos Tochter Ariadne hatte sich, als Theseus in Kreta war, in diesen verliebt, und ging mit ihm . Sie bekam von Dädalus einen Faden, der Ariadne
und Theseus aus dem Labyrinth heraus hielf. Zur Strafe sperrte Minos Dädalus mit seinem Sohn Ikarus in das Labyrinth ein. Dädalus kannte jedoch den
Ausgang, baute eine Art Flügel für sich und seine Sohn und floh so von Minos' Schloss nach Sizlien. Ikarus stürzte bei dem Flug jedoch ins Meer und
ertank. Minos verfolgte Dädalus bis nach Siziien, wo dieser bei König Kokalos Schutz gefunden hatte. Nachdem Minos Kokalos drohte, wollte Kokalos
Dädalus ausliefern, doch Minos wurde von Kokalos' wunderschönen Töchtern im Bade ertränkt.

 Auf Kreta gab es die Minoische Kultur, die erste frühe
Hochkultur der Welt. Bei dieser Hochkultur gab es bereits sehr füh spezialisierte Berufe wie Fischer, Ruderer, Kapitäne, Soldaten, Schreiber, Töpfer,
Maler, Bauarbeiter, Architekten und  Musiker. Hauptwerke der Kunst in der minoischen Kultur waren neben der Architektur auch die Wandmalerei und das
Malen auf Vasen. Die damaligen Bewohner Kretas sprachen die minoische Sprache schrieben mit Hieroglyphen. Es gab auch eine minoische Religion.  

\end{document}

